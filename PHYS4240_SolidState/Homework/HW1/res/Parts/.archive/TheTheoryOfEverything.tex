\section{\textit{The Theory of Everything} by Robert B. Laughlin and David Pines}

This article discusses the same core ideas as does \textit{More is Different}, albeit in a slightly more straightforward and less philosophical way. The authors start by presenting the ``Theory of Everything'', which is just a Hamiltonian that takes into account the kinetic and potential energies associated with the interactions among the constituents of atoms. Sure enough, for a system of just a handful of particles in an adequately high energy scale, this equation works quite well. However the core of the issue arises when considering any system of over 10 or so particles, which, in the grand scheme of things, is a remarkably small number. Such a calculation is both analytically and computationally impossible.

At this stage, attempting to describe a larger system with this fundamental ``Theory of Everything'' is completely inadequate and, ironically, the best attempts to use it to make any moderately useful predictions in larger scales require input from experiment done on these larger scales; nothing else is or can be derived from these fundamental laws. Essentially, this theory is completely useless on any other scale than that of a handful of particles.

Further, once one starts considering states the authors called \textit{quantum protectorates}, the observed phenomena and properties of matter at sufficiently low energy scales becomes regulated entirely and solely by larger organizational principles inherent to it at these energy scales. These princples are not at all apparent in any context where the ``Theory of Everything'' has any power. Further, these principles are apparent even in a classical scale, which leads to the conclusion that emergent phenomena may even be present in biological systems. Anderson in \textit{More is Different} speculated about some sort of time periodicity-like structure, such as how energy generators are cyclic, for instance, among several other examples.






%%% Local Variables:
%%% mode: LaTeX
%%% TeX-master: "../../HW1"
%%% End:

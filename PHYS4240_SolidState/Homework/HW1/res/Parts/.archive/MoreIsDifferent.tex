\section{\textit{More Is Different} by Philip W. Anderson}

The essential point that Anderson is arguing here is that the idea of \textit{constructionism} is not the logical next step to \textit{reductionism}. By this, he means that reductionism, the idea that we can continue subdividing problems and uncover laws that everything universally follows, does not \textit{imply} constructionism, the idea that from these ``fundamental'' laws we can fully explain all phenomena at any larger scale. Even as a particle physicist (to whom he directs negative comments on several occasions - but understandably so), I find that I agree with this general idea.

He approaches this from a few different lenses, progressively increasing the scale in terms of relative ``size'' of systems. He starts with the example of the ammonia molecule and how chemists say it has a dipole moment, but in reality, as a stationary state, parity symmetry dictates that it must be a superposition of all these asymmetric states that exhibit a dipole moment. At the end of the day, then, the stationary state does not have a dipole moment. The point he is getting at here is that at this scale, there is a parity symmetry present and dictating how the system behaves.

However, when we increase the scale by increasing the number of atoms present in molecules, it becomes less sensical to expect any sort of ``inversion'' between asymmetrical states, and on the scale of sugar molecules, it makes no sense at all, as sugar molecules are not expected to spontaneously invert like the ammonia molecules. At this stage, then, the whole idea of parity symmetry can be ignored. For this he adopted the term ``symmetry breaking'', which has no relation to spontaneous symmetry breaking as we know it in relation to the Higgs mechanism. What we are finding then is that the more ``fundamental'' symmetry that dictates the smaller system did not have any direct importance when examing, as a whole, the larger system. Though, it's not that the symmetry stops happening on the smaller scale, or that it suddenly isn't a symmetry at all, it just loses its power and becomes useless to study on the larger scale. Further, to tie into the main argument, by knowing that the parity symmetry exists on smaller scale led us in no way to any sort of discovery about the mechanics of the larger system.

We take yet another step up in scale to large crystals, and yet again find another sort of symmetry breaking. This time, inside of the crystal, there are possibly sections in which there does exist a net dipole moment, and yet these crystals don't have to be the result of the influence of living organisms. Indeed, at such a scale, it is nonsensical to suggest that the entire system should spontaneously invert itself due to quantum mechanical tunneling to its mirror state in any finite time, leading to, like with the ammonia molecule, a stationary state. We have yet another broken symmetry present here.

We can now start to take a step back and draw some conclusions, or more accurately, inferences. First, though, is the question of what truly is ``fundamental'' research: is it the research that led to the understanding of the parity symmetry on the scale of ammonia molecules? Is it the research on synthesized sugar molecules whose asymmetry is perhaps a fundamental component of how living organisms work? What about that of the crystal, where we have looped back to some sort of new symmetry within the crystals themselves that strictly does \textit{not} manifest in smaller systems? To be sure, the underlying parity symmetry has some governance over how the larger systems behave -- again, this symmetry doesn't just \textit{stop} happening among its constituents, but rather the system as a whole does not exhibit this symmetry anymore (at least in a finite time). A new consideration, that being the fact that the crystal seeks one such state due to it being the lowest energy among other states, arises. So, then, if such a observation does not seem to be relevent in examining smaller systems (at least in the same way), who is to say that this research is any less \textit{fundamental} than the research done by the atomic physicist on the ammonia molecule?

As another inference, the author brings up the idea that it is nigh impossible to go from quantum mechanics and predict the fact that the ammonia molecule does what it does, because this requires that we first describe the ammonia molecule itself as an asymmetrical object which undergoes these inversion, a fact that is not at all clear since no quantum mechanical state should have such an asymmetrical structure. In the next scale up, it would be very hard to predict that a nucleus can be oblong or some other non-featureless shape despite having no net dipole moment in its stationary state. The fundamental idea at play here is not even one necessarily relating to physics: it becomes increasingly hard, if not impossible, to define concrete characteristics of a many-body system when there are so few individual bodies. This is the case of the nucleus, where it is almost, but not quite an $N \rightarrow \infty$ system. This begs the question: how can we use laws that describe, at most, a few particles and their interactions, to describe a system of hundreds, thousands, or more particles, where new properties that are inherent to the larger number of bodies manifest?

Lastly, in the super macroscopic realm, matter undergoes phase transitions in which even more of the underlying symmetries are broken. Like with crystals, the system as a whole undergoes what are called phase transitions, in which the constituents of the matter in question find it energetically favorable to remain in certain configuration which dont have to be (and largely are not) symmetrical in the same was that a quantum mechanical state would be. This allows the object as a whole to change in response to some external force, such as a crystal's sections that have dipole moments can flip in response to an external electric field. This natural, energetically favorable state is one that defines a sort of ``rigidity'', a characteristic of some solid materials as well as, interestingly, superconductors and even superfluids. As if the point wasn't clear at this point, such emergent phenomena is not something that is present in higher energy scales.

It isn't an unreasonable assumption to say that this pattern of phenomena emerging at larger and larger scales should continue. Anderson goes on to speculate a number of such phenomena and give some reasoning why it may be the case, but I think the main point has been argued very well.





%%% Local Variables:
%%% mode: LaTeX
%%% TeX-master: "../../HW1"
%%% End:

\section{Author Information}


\subsection{Philip W. Anderson}

Philip Warren Anderson was a theoretical physicist who made a number of incredibly important contributions to the field of condensed matter physics, as well as paving the way to the understanding of the Higgs mechanism in high energy physics. Further, he is credited with the term \textit{emergence}/\textit{emergent phenomena} within the realm of philisophical science. This was most famously captured in an article named ``More is Different'' written in 1972. He was awarded the Nobel Prize in Physics in 1977 for ``fundamental theoretical investigations of the electronic structure of magnetic and disordered systems'', which significantly aided the development of components for computers.

Much to my dismay as a particle physicist, I have also learned that he advocated against the construction of the Superconducting Super Collider, which would have been a $\qty{40}{\tera\electronvolt}$ proton-proton collider in Texas.




\subsection{Robert B. Laughlin}

Robert Betts Laughlin is a theoretical physicist and a professor at Stanford University. He, along with Horst Störmer and Daniel C. Tsui, was awared the Nobel Prize in 1998 for work on the fractional quantum Hall effect (something which I won't attempt to try and figure out how to explain in my own words). Some of his current work at Stanford includes research into ``correlated-electron'' phenomenology, which implies that there is some sort of larger and new kind of quantum self-organization in materials. This idea of emergent phenomena was further explored in a book he authored by the name of \textit{A Different Universe}.




\subsection{D. Pines}

David Pines was a physicist who worked heavily in a number of fields, most notably in condensed matter and nuclear physics in many-body systems. He, along with other notable physicists at the time like David Bohm and John Bardeen, introduced quasiparticles like the plasmon and also paved the way for the development of the BCS theory of superconductivity. He also helped to organize a large number of workshops and summer schools in the US and abroad and was a member and fellow of many different organizations. Some of his more recent research before his death involved, like others in this assignment, the exploration of the idea of emergent phenomena in matter.


\subsection{Steven Weinberg}

Steven Weinberg was a theoretical physicist who was immensely impactful for his research in elementary particle physics, in which he won the Nobel Prize along with Abdus Salam and Sheldon Glashow for their development of the GWS theory of the weak force. He is often named as one of the most significant theoretical physicists of the 20th century for his significant contributions to quantum field theory. He is the author of the volumes titled \textit{The Quantum Theory of Fields}. Weinberg was also known to be a ``public spokesperson for science'' due to his various lectures and publications directed towards a more historical or philosophical point of view. He, unlike Anderson, was a proponent for the SCC!



\subsection{John J. Hopfield}

John Hopfield is a physicist who is most notably known for his research into neural networks and their applications for physics, for which he and Geoffrey Hinton were awarded the Nobel Prize last year (2024). Two of his most credited articles involved ``Hopfield networks'', a novel type of neural network, for which inspiration came from spin glass systems. He has also done work is a large number of other fields, such as in condensed matter in his earlier life, such as his PhD work in which he described (and coined the term) polaritons, quasiparticles in crystals. He was also considered Anderson's ``hidden collaborator'' as efforts were made to explain the Kondo effect.


\subsection{Nigel Goldenfeld}

Nigel Goldenfeld is a professor of physics at the University of California in San Diego. His work spans several fields, largely condensed matter, statistical physics, living systems, and hydrodynamics. His work in condensed matter physics led to developments for high temperature superconductors. He was also instrumental in the understanding of patterns in slowflakes, and consequently, pattern formation in general in nature. His biological interests were fruitful during the COVID-19 pandemic, as his research helped set up and run a COVID saliva testing system which boasted very quick and accurate results.



%%% Local Variables:
%%% mode: LaTeX
%%% TeX-master: "../../HW1"
%%% End:

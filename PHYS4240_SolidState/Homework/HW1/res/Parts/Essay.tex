\section{Essay}

I was originally going to do this article-by-article, but I found it harder to analyze one specific topic while also including enough of my own input. For instance, it was quite challenging to try and write an analysis on just \textit{More is Different} by Anderson, since he has one set of things he talks about and a path he takes to his one main argument. The only way I could really provide an analysis that was substantially my own was by bringing in outside stuff, which, by choosing to analyze just the one article, wasn't easy. Further, I don't really understand much physics related to solid state compared to particle physics (yet, at least), and this made it even more challenging. Instead, then, I'll just write my own one of the these articles, sort of, and my goal will be just to share my thoughts on the subject.

This entire topic is particularly interesting to me as a high energy physicist, since my main philosophy is inherently reductionist. Of course, even before reading these articles I knew that, in some capacity, diving into higher and higher energies revealed increasingly little about real-world experiences. As an example, I was able to do predictions that (roughly) matched experiment in my high school chemistry class without needing to even know what a quark is, let alone the pandora's box that is QCD. Even though a proton is a bound state of a bunch of quarks and gluons and atomic properties are governed in part by the number of these protons in the atom, these laws/symmetries of QCD played absolutely no role in the predictions made. Of course, these predictions were never perfect (I clearly remember getting something near a 40\% error in one instance), but in principle, a perfect experimental procedure would be sure to produce results that agreed well enough with predictions.

Another example of something even more pertinent to everyday life is the simple act of dropping a ball. The ball's mass along with the Earth's mass create a mutual attraction that brings them together. The Earth is enormously heavy, so it hardly moves at all, while the ball accelerates at $\sim\qty{9.8}{\meter\per\second}$ towards the ground. I remember running an experiment of dropping a ball a bunch of times from different heights and using the slow-mo feature on my phone's camera to determine how long each trip took, and then determining the actual numerical value of the acceleration the ball experienced. I never needed to know anything but the simple idea that force and acceleration are directly proportional to mass, a fact developed centuries ago. However, as particle physicists we know that mass is granted to the fundamental particles constituting the ball via the Higgs mechanism. And yet, I didn't even know about this phenomena, much less about anything smaller than an atom, at this point in my education.

However, QCD still happens, the Higgs mechanism still happens, all of the things high energy has discovered still happen on their respective scales, it's not like these things \textit{stop} happening once we start zooming out. My understanding of emergence and ``more is different'' is that these symmetries simply stop being relavant once we zoom out. Specifically, when we do zoom out and examine a large number of particles that are themselves goverened by these symmetries, the system as a hole no longer does -- it gains its own unique type of symmetry that is only apparent, and in some cases \textit{inherent} to large systems.

There was one concise example from the reading that I was able to understand best that made this idea clear to me. This was the idea of spontaneous magnetization, something I looked at when studying the Ising Model for my Computational Physics I project. Zooming in to a magnet, and examining the atoms themselves, we note that the laws governing the atom are rotationally invariant; it doesn't matter how the atom is oriented, it'll still work the same way. However, when you look at the magnet as a whole and lower the temperature a bunch, what we find is that the individual atoms choose to magnetize in one particular direction. This creates a magnetic field, and hence, the laws governing the magnet as a whole now clearly no longer exhibit the three-dimensional rotational symmetry like that of its consituents.

Further, this state stays this way. Unlike the ammonia atom, something discussed by Anderson, such a large structure is in no way expected to spontaneously invert itself via quantum mechanial tunneling or any other phenomena in any finite amount of time. This is simply the lowest preferred energy state of the magnet. And again, it's not like the individual atoms themselves stop having a three-dimensional rotational symmetry, it's simply that by examining the magnet as a large collection of atoms, that symmetry has been ``broken'', and it is no longer obeyed by the composite system. A new type of symmetry is apparent, or, in other words, \textit{emerges}.




%%% Local Variables:
%%% mode: LaTeX
%%% TeX-master: "../../HW1"
%%% End:

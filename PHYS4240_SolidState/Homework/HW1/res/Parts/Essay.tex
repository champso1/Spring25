\section{Essay}

I was originally going to do this article-by-article, but I found it harder to analyze one specific topic while also including enough of my own input. Each article has a main point and a main path of argumentation used to get to that point; they were each self-contained, in a way. The only way I could really provide an analysis that was substantially my own was by bringing in outside stuff, which, by choosing to analyze just the one article at a time, wasn't easy. Further, I don't really understand much physics related to solid state compared to particle physics, and this made it even more challenging. Instead, then, I'll just write my own one of the these articles, sort of, and my goal will be just to share my thoughts on the subject, and talk about things that I want to talk about.

This entire topic is particularly interesting to me as a high energy physicist, since my main philosophy is inherently reductionist. Of course, even before reading these articles I knew that, in some capacity, diving into higher and higher energies revealed increasingly little about real-world experiences. As an example, I was able to do predictions that (roughly) matched experiment in my high school chemistry class without needing to even know what a quark is, let alone the pandora's box that is QCD. Even though a proton is a bound state of a bunch of quarks and gluons and atomic properties are governed in part by the number of these protons in the atom, these laws/symmetries of QCD played absolutely no role in the predictions made. It should be said, though, that these predictions were never perfect; I clearly remember getting something near a 40\% error in one instance, but in principle, a perfect experimental procedure would be sure to produce results that agreed well enough with predictions.

Another example of something even more pertinent to everyday life is the simple act of dropping a ball. The ball's mass along with the Earth's mass create a mutual attraction that brings them together. The Earth is enormously heavy, so it hardly moves at all, while the ball accelerates at $\sim\qty{9.8}{\meter\per\second}$ towards the ground. I remember running an experiment of dropping a ball a bunch of times from different heights and using the slow-mo feature on my phone's camera to determine how long each trip took, and then determining the actual numerical value of the acceleration the ball experienced. I never needed to know anything but the simple idea that force and acceleration are directly proportional to mass, a fact developed centuries ago. However, as particle physicists we know that mass is granted to the fundamental particles constituting the ball via the Higgs mechanism. And yet, I didn't even know about this phenomena, much less about anything smaller than an atom, at this point in my education. As I came to understand more and more about physics and about things that are more ``fundamental'' it didn't reveal suddenly that the ball should drop at any different speed than what we observed, or that chemical reactions should proceed any differently than how they did during our experiments.

However, despite these things having no apparent effect in our scale (by which I mean the symmetries and physical laws governing these fundamental systems not being the same symmetries that govern us at our scale), QCD still happens, the Higgs mechanism still happens, all of the things high energy has discovered still happen on their respective scales -- it's not like these things \textit{stop} happening once we start zooming out. My understanding of emergence and ``more is different'' is that these symmetries simply stop being \textit{relevant} once we zoom out. Specifically, when we do zoom out and examine a large number of particles that are themselves goverened by these symmetries, the system as a hole no longer does, rather it gains its own unique type of symmetry that is only apparent, and in some cases \textit{inherent}, to large systems.

There was one concise example from the reading that I was able to understand best that made this idea clear to me. This was the idea of spontaneous magnetization, something I looked at when studying the Ising Model for my Computational Physics I project. Zooming in to a magnet, and examining the atoms themselves, we note that the laws governing the atom are rotationally invariant; it doesn't matter how the atom is oriented, it'll still work the same way. However, when you look at the magnet as a whole -- and lower the temperature a bunch -- what we find is that the individual atoms decide to align/magnetize in one particular direction. This creates a magnetic field, and hence, the laws governing the magnet as a whole now clearly no longer exhibit the three-dimensional rotational symmetry like that of its consituent particles.

Further, this state stays this way. Unlike the ammonia atom, something discussed by Anderson, such a large structure is in no way expected to spontaneously invert itself via quantum mechanical tunneling or any other phenomenon in any finite amount of time. This is simply the lowest preferred energy state of the magnet. By some property inherent only to systems of a sufficiently large number of particles, the underlying three-dimensional rotiational symmetry has been ``broken,'' and no longer governs the state of the system as a whole. This behavior, where new symmetries spring forth and old symmetries die out at a larger scale is what I understand \textit{emergence} to be.

Back to my main point I made near the beginning of this essay: all of this is, without a doubt, true, and I never believed that by diving deeper and deeper with higher and higher energies we'd eventually just find the answer to everything at every energy scale. As I've just discussed, new behavior only inherent to larger systems emerges at their corresponding length/energy scales. Despite this, though, I don't necessarily think there is any less merit to studying high energy, or that there is anything wrong with a reductionist point of view, provided that the main goal isn't an attempt to create a Theory of Everything, or an attempt to claim it as any more ``fundamental'' than any other field.

Additionally, I believe that there is simply a different motivation/curiosity between high energy and non-high energy. To explain what I mean by this, let's consider starting at the atomic scale. There are two different questions that can be asked, one that requires higher energy and one that requires a lower energy. One such question of the former type is simply: what is inside the proton? A question of the latter type is (also simply): what happens when we stick a bunch of these atoms together? Of course, it is easy to tell that one is inherently reductionist and the other isn't, but in my opinion, there is more to it. Both viewpoints are certainly motivated by curiosity, but I think that there is a different type of underlying curiosity motivating each question.

I've found it very hard to put into words exactly why I think this way. The best I've found is like so: just as John Hopefield talked about in the first couple of paragraphs, I was similarly mesmerized with a great many things I was surrounded by when I was younger. I had a small bias towards things like computers and related things, but that was just because my father is an electrical engineer, so we had a lot of that stuff around. But I think this is where the curiosity differs. My thinking led me to, think \textit{well, I've taken apart the computer and I can see how this all works, but how does each component work?} Once I found out about transistors, I wanted to learn how those worked, and what constituted semiconductors, and so on and so on until I reach the atom, where I then asked the reductionist question I posed in the previous paragraph, and kept going and going.

I'm not a solid state physicist, and, as I've mentioned a number of times, my mindset is fundamentally different from one, but I feel like physicists in that field stop at some point, and start looking at the bigger picture and instead of asking what a particular thing is made of, they ask what they can do with the things they have, and what happens when you start putting these together in certain ways. Going back to the computer analogy, a solid state physicist would stop at the transistor step, and ask what you could make out of the transistors, and what would happen if, for instance, you combine a bunch of them together. Perhaps some emergent phenomena, like a CPU, pops out once you put enough of them together! Obviously the solid state physicist would be very interested in the semi conductors within the transistors, but I think the analogy still works.

There is another thing about emergence that I want to give my opinions and thoughts on, and that is conciousness, the ultimate emergent phenomena. I also will touch on other possible emergent phenomena, simular to how Anderson speculated a bit in the final few paragraphs of his article. In terms of conciousness, no one has any idea how it works, and I think that is such a fascinating thing. If I wasn't going to be a physicist, I would consider a psychological field, since the brain provides an infinite number of questions to which we have approximately zero (concrete) answers. This is not to say that the brain is the only thing that humans don't quite understand, but many such things, at least in physics, are exotic phenomena that ordinary people don't experience on a day-to-day basis. In principle, the entirety of high energy physics. But the brain is the exact opposite; it is the thing that \textit{experiences}. It is the thing that \textit{does} anything for us in the first place. It sends the electrical signals throughout our body and commands the muscles in the mouth and throat to say ``we need another particle collider with even higher energy!'' among many other things. But why? In particular, why do only some of us have our brain send those signals and have us say those things? Even more in particular, why do some of us \textit{think} one way or the other? I could keep on going, but I think the point has been made.

The entire idea of thinking in general is such a fundamental concept to us since we think everyday (well, some of us do), but if you take a minute to step back and, for lack of a better work, think about what thinking \textit{is}, it doesn't really make any sense at all. Our brains are a perfectly reasonable number of neurons - compared to the number of particles non-high energy physicists deal with, for instance - yet some completely unfathomable level of emergence takes form and gives us what we dub conciousness. If such a phenomenon exists on the scale of our brain with so few answers, who's to say that there isn't some other sort of new symmetry that emerges on the scale of the earth, with its billions of people? What about the universe? Importantly, all of these questions are fundamental, in my eyes, since they each tackle unique problems that appear on that scale and no other scale (assuming such problems appear in the first place).

I do wonder what the future of physics will look like in, say, 100 more years. Perhaps we will discover another set of truly fundamental particles to add to the repertoire of the Standard Model. Perhaps more developments will be made on the other side of the energy spectrum, and we may have room-temperature superconductors. Or, maybe we will have uncovered something fundamental about the human consciousness! Potentially after another thousand years, we will find that there is some higher symmetry connecting all life on Earth... at what point do we circle back to religion? I want to say that there can't possibly be such a symmetry, but with us having such a complete lack of understanding of what is inside or own heads, you never know, it's something that certainly is possible!

With this, I think I have exhausted what I can discuss about this topic without having anything be forced. I am well aware that I didn't get to the word count (though I don't know if the word count was for just the essay or for the bibliographies and new concepts sections), and I wasn't even really all that close, either. I could have also analyzed each article one by one as well, but this would have been disjointed, and not particularly illuminating since, like I mentioned in the beginning of this essay, there's only so much information I can add myself without things being forced or having to do a bunch of research, which I don't quite have time for.



%%% Local Variables:
%%% mode: LaTeX
%%% TeX-master: "../../HW1"
%%% End:

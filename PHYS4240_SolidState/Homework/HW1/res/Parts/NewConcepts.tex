\section{New Concepts Encountered}

There were a plethora of concepts I found that I didn't understand, and it would've been a lot to include every one of them, specifically some of highly specific ones or, on the other hand, highly general things that can't be represented by a single term -- I selected only a handful that did not fit into this category and that were interesting. For some technical terms, I don't hesitate to copy from Wikipedia for a general definition.


\begin{itemize}
  
\item \textbf{Pyroelectricity}: This was mentioned in Anderson's ``More is Different'' along side ferooelectricity, which I was far more familiar with from my Computational Physics I project, but I hadn't heard of pyroelectricity before. It is a property that certain materials may have in which they simply are naturally electrically polarized. Just like with ferro\textit{magnets}, in which the direction of magnetization can be reversed with an external magnetic field, ferroelectric materials can have these polarization reversed from an external electric field. Pyroelectric materials don't have this property, but it can be influenced by temperature changes, as the atoms within the crystal structure rearrange, which can shift the polarization.
  
\item \textbf{Josephson Effect}: This is an example of quantum mechanical effects occurring at normal scales, which is super crazy. Essentially, when you place two superconductors super close together separated only by a thin barrier, it turns out that it produces something called a ``supercurrent'', an electric current that flows freely within a super conductor, without any applied voltage. Interestingly, this phenomena had been observed before Josephson did (and coined it), but it was attributed to experimental error. Nowadays, it has a number of implications quantum-mechanical circuits.
  
\item \textbf{(Fractional/Integer) (Quantum) Hall Effect}: The normal Hall effect is a macroscopic phenomenon where if there is an electrical current traveling through a conductor and a magnetic field going perpendicularly to the it, then there is a produced potential difference within the conductor transverse to the current. Going to the quantum scale and considering two-dimensional electron systems, if we subject them to low temperatures and strong magnetic fields, we observe something like the Hall effect in that the resistance is \textit{quantized} and proportional to Planck's constant and inversely to the electron charge squared times an integer, or in the case of the fractional quantum Hall effect, times a fractional value. Though, in this latter case, it is apparently significantly more complicated and is an active area of research.
  
\item \textbf{(Landau's )Fermi Liquid Theory}: There is a lot to unpack while trying to look at this concept, so I'll only broadly define it to the best of my ability. This theory attempts to form a model of interacting fermions within many-body systems where these interactions may be strong. There are connections drawn between Fermi liquids and Fermi gases, which are idealizes collections of \textit{non-}interacting particles. This theory is most applicable to conduction electrons in metals as well as liquid helium-3, for instance.

\item \textbf{Kondo Insulators}: This one is also a bit complicated, so I'll do my best. I picked this term since it evidently is related to the Kondo problem, which I may consider for my project. A Kondo insulator is a material with strongly correlated electrons such that a band gap opens at low temperatures. In this band gap is the chemical potential. Ordinarily, the chemical potential lives in the conduction band.

\item \textbf{Superfluidity/Liquid Helium-3/4}: Liquid helium-3 and 4 are brought up quite a number of times in the texts, and they are related to superfluidity. When temperatures are brought low enough for a helium-3 state, as fermions, they form Cooper pairs, a bound state of electrons at low temperatures. These then subsequently condense into a superfluid. Helium-4 is a bosonic state, so it obeys Bose-Einstein statistics and the constituent atoms are allowed to occupy the lowest energy state automatically, forming a Bose-Einstein condensate. A superfluid has some funky properties, one of which is that it has zero viscocity. It can ``creep'' along surfaces in order to level itself if placed in two adjacent containers.

\item \textbf{BCS Theory of Superconductivity}: This one is definitely a lot to unpack, I'll try and keep it brief. the Bardeen-Cooper-Schrieffer (BCS) theory of superconductivity is the first microscopic theory of superconductivity. The essential idea at play is similar to how liquid helium-3 becomes a superfluid, by which at low temperatures the constituents form Cooper pairs, which then condense. The three physicists involved received the Nobel Prive in 1972.

\end{itemize}

Again, of course there were a number of other terms that I didn't quite understand, but I felt that they were either far too technical or far to general, like someone's specific thesis or just ``superconductivity.'' I did include the BCS theory of superconductivity because that is a specific theory (and a monumental one).


%%% Local Variables:
%%% mode: LaTeX
%%% TeX-master: "../../HW1"
%%% End:

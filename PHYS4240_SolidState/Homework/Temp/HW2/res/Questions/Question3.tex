\section{The Harmonic Oscillator}

\begin{parts}
\item The ladder operators are defined like
  
  \begin{equation}
    \hat{a}_\pm \equiv \frac{1}{\sqrt{2m\omega\hbar}}(\mp i\hat{p} + m\omega\hat{x}).
  \end{equation}

  Writing the combination $\hat{a}_+\hat{a}_-$:

  \begin{align}
    \hat{a}_+\hat{a}_- &= \frac{1}{2m\omega\hbar}\left(\hat{p}^2 - im\omega\hat{p}\hat{x} + im\omega\hat{x}\hat{p} + (m\omega\hat{x})^2\right) \\
                       &= \frac{1}{2m\omega\hbar}\left(\hat{p}^2 + (m\omega\hat{x})^2 + im\omega[\hat{x},\hat{p}]\right).
  \end{align}

  The commutator $[\hat{x},\hat{p}] = i\hbar$, so

  \begin{equation}
    \hat{a}_+\hat{a}_-= \frac{1}{\omega\hbar}\left[\frac{1}{2m}(\hat{p}^2 + (m\omega\hat{x})^2)\right] + \frac{i}{2\hbar}(i\hbar).
  \end{equation}

  The first term is just the Hamiltonian, and the second is $-1/2$:

  \begin{gather}
    \hat{a}_+\hat{a}_- = \frac{1}{\hbar\omega}\hat{H} - \frac{1}{2}, \\
    \boxed{\hat{H} = \hbar\omega\left(\hat{a}_+\hat{a}_- + \frac{1}{2}\right).}
  \end{gather}

  Or, we also associate the ``raising'' operator with the ``creation'' operator which we typically denote $\hat{a}^{\dagger}$, where the annihilation operator is its Hermitian conjugate $\hat{a}$, so our Hamiltonian can also be written like so:

  \begin{equation}
    \boxed{\hat{H} = \hbar\omega\left(\hat{a}^{\dagger}\hat{a} + \frac{1}{2}\right).}
  \end{equation}



\item The number operator is defined as $\hat{N} \equiv \hat{a}^{\dagger}\hat{a}$, meaning we could write the Hamiltonian yet another way:

  \begin{equation}
    \hat{H} = \hbar\omega\left(\hat{N} + \frac{1}{2}\right).
  \end{equation}

  The Hamiltonian is now constructed with the number operator and constant terms. Since the number operator obviously commutes with itself and any constant, it must also commute with the Hamiltonian. Therefore, there must be a set of simultaneous eigenstates between them.

\item With the Hamiltonian written with the number operator, the eigenvalues of the kets $\ket{n}$ are trivial:

  \begin{equation}
    \hat{H}\ket{n} = \hbar\omega\left(\hat{a}^{\dagger}\hat{a} + \frac{1}{2}\right)\ket{n} = \hbar\omega\left(n + \frac{1}{2}\right)\ket{n} = E_n\ket{n},
  \end{equation}

  so $E_n = \hbar\omega(n + \frac{1}{2})$.


\item Let's consider $\hat{a}\ket{n} = A_n\ket{n-1}$, where $A_n$ is the eigenvalue we are trying to find. We know that if we create a sandwich with the number operator:

  \begin{equation}
    \braket{n | \hat{a}^\dagger \hat{a} | n} \rightarrow \braket{\hat{a}n | \hat{a}n} = \abs{A_n}^2\braket{n-1 | n-1}.
  \end{equation}

  However, obviously, since it's the number operator, we also have

  \begin{equation}
    \braket{n | \hat{a}^\dagger \hat{a} | n} = n\braket{n | n},
  \end{equation}

  meaning we have

  \begin{equation}
    \abs{A_n}^2\braket{n-1 | n-1} = n\braket{n | n}.
  \end{equation}

  Via Dirac orthonormality, this becomes $\abs{A_n}^2 = n$, so $A_n = \sqrt{n}$ since $n$ is positive/real.

  Next, we consider $\hat{a}^\dagger \ket{n} = B_n\ket{n+1}$. In this case, we make the sandwich $\braket{n | \hat{a}\hat{a}^\dagger | n}$, which is

  \begin{equation}
    \braket{n | \hat{a}\hat{a}^\dagger | n} = \braket{\hat{a}^\dagger n | \hat{a}^\dagger n} = \abs{B_n}^2\braket{n+1 | n+1}.
  \end{equation}

  Now, since $[\hat{a}, \hat{a}^\dagger] = 1$, this is also

  \begin{equation}
    \braket{n | \hat{a}\hat{a}^\dagger | n} = \braket{n | \hat{N} + 1 | n} = (n+1)\braket{n | n}.
  \end{equation}

  Therefore, $B_n = \sqrt{n+1}$, so

  \begin{align}
    \Aboxed{\hat{a}\ket{n} &= \sqrt{n}\ket{n-1},} \quad\text{and} \\
    \Aboxed{\hat{a}^\dagger \ket{n} &= \sqrt{n+1}\ket{n+1}.}
  \end{align}


\item Writing a few terms, we find

  \begin{align}
    \ket{1} &= \frac{1}{\sqrt{1}}\hat{a}^\dagger \ket{0}, \\
    \ket{2} &= \frac{1}{\sqrt{2}}\hat{a}^\dagger \ket{1} = \frac{1}{\sqrt{2 \cdot 1}}(\hat{a}^\dagger)^2\ket{0}. \\ 
    \ket{2} &= \frac{1}{\sqrt{3}}\hat{a}^\dagger \ket{2} = \frac{1}{\sqrt{3 \cdot 2 \cdot 1}}(\hat{a}^\dagger)^3\ket{0},
  \end{align}

  and so on. Clearly, this is just

  \begin{equation}
    \boxed{\ket{n} = \frac{1}{\sqrt{n!}}(\hat{a}^\dagger)^n\ket{0}.}
  \end{equation}

  
\end{parts}

%%% Local Variables:
%%% mode: LaTeX
%%% TeX-master: "../../HW2"
%%% End:

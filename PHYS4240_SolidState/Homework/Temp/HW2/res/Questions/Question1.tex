\section{The Delta Function Potential}

I have no idea how much you want us to show, so I showed quite a bit just in case!

In the case of the delta function potential, the case of interest for this problem is scattering states where $E > 0$. With this, the Hamiltonian for a particle of mass $m$ is

\begin{equation}
  \hat{H} = \frac{\hat{p}^2}{2m} + \lambda\delta(x).
\end{equation}

In the regions $x < 0$ and $x > 0$, the potential term vanishes due to the delta function, and we are left with:

\begin{equation}
  \hat{H} = -\frac{\hbar^2}{2m} \diff[2]{}{x},
\end{equation}

so the Schr\"odinger equation looks like

\begin{equation}
  \hat{H}\psi(x) = E\psi(x) \quad\rightarrow\quad -\frac{\hbar^2}{2m} \diff[2]{\psi(x)}{x} = E\psi(x).
\end{equation}

This is something we have seen on a number of occasions. Defining $k \equiv \sqrt{2mE}/\hbar$, our solutions are complex exponentials:

\begin{equation}
  \psi(x) =
  \begin{alignedat}{2}
  \begin{cases}
    Ae^{ikx} + Be^{-ikx} \quad & x < 0, \\
    Ce^{ikx} + De^{-ikx} \quad & x > 0.
  \end{cases}
  \end{alignedat}
\end{equation}

The usual boundary conditions at $\pm\infty$ don't apply here, i.e. we can't eliminate one term due to it blowing up, since these are complex exponentials. But, intuitively, since there is no other boundary past the delta function potential, we would expect that the term proportional to $D$, involving a wave propagating from the right on the right side of the potential, should be zero. Thus $D = 0$.

The next boundary condition we can consider is imposing that $\psi(x)$ be continuous at $x = 0$, meaning that we have

\begin{equation}
  A + B = C.
\end{equation}

The last condition we can consider is that the first derivative of $\psi(x)$ must also be continuous at $x=0$, except at points where the potential is not, which is exactly what we have here. In this case, we can use the prescription that

\begin{equation}
  \Delta\left(\diff{\psi}{x}\right) = \lim_{\epsilon \rightarrow 0} \left(\diff{\psi}{x}\bigg|_{+\epsilon} - \diff{\psi}{x}\bigg|_{-\epsilon}\right) = \frac{2m}{\hbar^2} \lim_{\epsilon \rightarrow 0} \int_{-\epsilon}^{+\epsilon} V(x)\psi(x) \;\dd x.
\end{equation}

For us, since we have a delta function, we don't need to actually do the integral and we have

\begin{equation}
  \Delta\left(\diff{\psi}{x}\right) = \frac{2m\lambda}{\hbar^2}\psi(0). 
\end{equation}

Now,

\begin{gather}
  \diff{\psi}{x}\bigg|_{-\epsilon} = \left[ik(Ae^{ikx} - Be^{-ikx})\right]_{-\epsilon} = ik(A - B), \\
  \diff{\psi}{x}\bigg|_{\epsilon} = \left[ik(Ce^{ikx})\right]_{\epsilon} = ikC, \\
\end{gather}

so

\begin{equation}
  ik(A - B - C) = \frac{2m\lambda}{\hbar^2}(A + B).
\end{equation}

Doing some rearranging, we find

\begin{equation}
  A\left(1 + 2i \frac{m\lambda}{\hbar^2k}\right) - B\left(1 - 2i \frac{m\lambda}{\hbar^2k}\right) = C.
\end{equation}

Defining $\alpha \equiv \hbar^2/(m\lambda)$, we can say

\begin{equation}
  A\left(1 + \frac{2i}{k\alpha}\right) - B\left(1 - \frac{2i}{k\alpha}\right) = C.
\end{equation}

I'll leave out the tedious algebra; we find that

\begin{equation}
  B = \frac{i/k\alpha}{1 - (i/k\alpha)}A, \quad\text{and}\quad C = \frac{1}{1 - (i/k\alpha)}.
\end{equation}

Intuitively, based on the fact that the wavefunction is normalized, the coefficients are almost like probabilities, or relative sizes of the corresponding propagating waves, so we can treat their ratios with $A$ to determine relative probabilities that the forward propagating wave either transmits or reflects. For the former case, this is

\begin{equation}
  T = \frac{\abs{C}^2}{\abs{A}^2} = \frac{1}{1 + 1/(k\alpha)^2} = \boxed{\frac{(k\alpha)^2}{1 + (k\alpha)^2},}
\end{equation}

and for the latter case, this is

\begin{equation}
  R = \frac{\abs{B}^2}{\abs{A}^2} = \frac{1/(k\alpha)^2}{1 + 1/(k\alpha)^2} = \boxed{\frac{1}{1 + (k\alpha)^2}.}
\end{equation}

The determine the unit of $k$, we consider that the quantity in an exponential must be unitless, so the units of $k$ and $x$ must cancel, so $[k] = \unit{\per\meter}$ (meters). The transmission/reflection coefficients must also be unitless, since all ratios are, meaning that the units of $\alpha$ are the inverse of $k$, so $[\alpha] = m$. Lastly, since the units of potential must be the same as those of energy, then $\lambda$ has units of energy.



%%% Local Variables:
%%% mode: LaTeX
%%% TeX-master: "../../HW2"
%%% End:

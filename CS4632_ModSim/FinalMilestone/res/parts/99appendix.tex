\section{ColSim Configuration File}\label{sec:colsim-config-file}

\begin{listing}[ht!]
\begin{minted}{text}
# -----------------------------
# ---- General Information ----
# -----------------------------
# pdf set to use
PDFName=CT18NNLO

# ---------------------------
# ----- Hard Scattering -----
# ---------------------------
# number of evaluations of the differential cross section in the Monte Carlo integration
NumXSIterations=1000000

# Center-of-Mass energy: measured in TeV
ECM=14.0

# the lowest energy used in calculating the cross section
MinCutoffEnergy=60.0

# these are intermediate calculated parameters used in a change-of-variables transformation
# during the cross section calculation -- there is no real physical significance
TransformationEnergy=60.0

# --------------------------
# ---- Parton Showering ----
# --------------------------

# the initial energy that the quark carries before starting evolution (in GeV)
# usually set to be characteristic of a proton-proton subprocess, ~1 GeV
InitialEvolEnergy=1000.0

# whether to fix the factorization scale throughout the
# evolution or vary it with the evolution scale
# NOTE: fixing the scale misses out on some higher order effects
# Yes or No
FixedScale=Yes

# the energy cutoff reached before evolution stops (in Gev)
# this is the energy scale in which perturbative QCD stops being valid
# cannot go lower than 1.0
EvolutionEnergyCutoff=1.0
\end{minted}
\caption{\textsc{ColSim}'s default configuration file, providing the default values for all input parameters.}
\label{listing:colsim-config-file}
\end{listing}



\pagebreak

\section{Listing for Gnuplot Test}\label{sec:gnuplot-listing}

\begin{listing}[ht!]
\begin{minted}{cpp}
// required for setup of random number generator
ColSimMain _;

// create gnuplot object
Gnuplot plot;

// specify plot information
vector<double> min{0.0}, max{3.14}, delta{3.14};
double numBins = 25;
plot.setHistInfo(min, max, delta, numBins);

// give col/file name, open the datafile
vector<string> colNames{"sin2_x"};
plot.openDataFile("data.dat", colNames);

// grab values of sin(x)
// in general we take a vector of vectors
// for multiple variables,
// that's the reason for adding a vector
uint numPoints = 0;
while (numPoints < 100000) {
	double x = randDouble() * 3.14;
	double r = randDouble();
	double val = sin(x)*sin(x);
	double ratio =  val / 1.0;
	if (r < ratio) {
	    plot.addDataPoint(vector<double>{x});
		numPoints++;
	}
}
	
// define the x and y labels
vector<string> xlabel{"x"}, ylabel{"sin^2(x)"}, title{"Histogram format for sin^2(x)"};
plot.setTitles(title);
plot.setXLabels(xlabel);
plot.setYLabels(ylabel);

// make the plots
plot.plot();
\end{minted}
\caption{Listing for the plotting of $\sin^2(x)$. Given without a main function or includes.}
\label{listing:gnuplot-testing}
\end{listing}


%%% Local Variables:
%%% mode: LaTeX
%%% TeX-master: "../../FinalMilestone"
%%% End:

\section{Introduction}

In Sec.~\ref{sec:PP-Collisions} and Sec.~\ref{sec:Sim-Considerations} I briefly describe the motivation for wanting to simulate proton-proton collisions and the theory behind the considerations that must be taken to model such a process. In Sec.~\ref{sec:Method-Implem}, I describe the methodology and implementation strategies I plan to use, including the programming language of choice, external tools, as well as a very brief overview of how I plan to implement the model in code. In Sec.~\ref{sec:Timeline}, I roughly estimate the timeline of when parts of this project are completed.


\subsection{Proton-Proton Collisions}\label{sec:PP-Collisions}

The goal of high-energy/particle physics is to try and understand how the universe works on a fundamental scale. One of the main ways we do this is by accelerating protons to near the speed of light, and colliding them together. This is done at CERN in Geneva, Switzerland, in what is called the Large Hadron Collider (LHC). The purpose of doing these extremely energetic collisions is to try and analyze the thousands of particles that fly out in all different directions and their subsequent decays and interactions to try and gauge what exactly happened in the collision. By analyzing the final-state stable particles, we can reconstruct various quantities and make plots and other predictions. As an example, if a heavy particle decays into two new, lighter, and more stable particles, we can analyze their energy and their paths once they reach the detector and determine (roughly) where/when the particle decayed and how massive it was. This is the essential recipe for discovering new particles, which is one of the main goals of the LHC.

It is often extremely helpful to have a theoretical framework that is able to match what is seen in the detectors so that we can make predictions with that framework in the same as well as adjacent contexts. This calls for the usage of complex simulation programs that model the entire chain of events starting from the ``hard scattering'' process, which is the initial collision, all the way to simulating the detector structure and how it behaves when the particles from the collision fly through it using this theoretical framework. We then read the output that the simulated detector shows us, and we can analyze the exact same things as done in experiment to make very accurate comparisons and predictions.


\subsection{Simulation Considerations}\label{sec:Sim-Considerations}

As one would expect, though, simulating these processes is immensely challenging. It is not simply a matter of momentum/energy conservation and other simple properties; there are a multitude of quantum mechanical phenomena that occur on this energy scale, the most notable of which come from Quantum Chromodynamics (QCD). Essentially, the constituents of protons, which are called \textit{quarks}, experience a force similar to that of normal electromagnetism, but opposite in some respects. For instance, bringing two electrons closer together increases the repulsion between them, since like charges repel. Similarly, bringing a proton and an electron closer together will increase the attraction between the two. Quarks, on the other hand, have the opposite property, where the closer you bring them together the less they feel like doing anything at all, and the further you bring them apart, the stronger the force between them becomes. This force is so strong that they actually cannot exist on their own: if you try and break two of them apart, at some point the energy required will be so high that two entirely new quarks will spontaneously form and bind with the two existing quarks and create new states called \textit{hadrons}, of which protons and neutrons are examples.

These effects manifest during a collision, where quarks will fly apart due to the immense amount of energy used to accelerate the protons but due to the nature of their interactions via QCD, once they get far enough apart, they will \textit{hadronize}, by which two new hadrons are formed. These can further collide with other quarks or hadrons, and also decay into other particles (potentially quarks again!) if they are unstable. This creates a multitude of essentially simultaneous processes that must all be considered. The name given to this part of the simulation after the hard scattering is called ``parton showering'' or ``hadronization.''

To make matters worse, in particle physics (and often physics in general), we will never be able to calculate the full answer to any given problem. This isn't an exaggeration; it's not just that the equations are super long (see \cite{MATHEWS2005333} for instance), it's that the best way we know how to formulate things is in terms of an \textit{infinite} series of terms. In most cases, fortunately, these terms get significantly smaller to the point that we can get extremely reasonable estimates after only 3 terms. So, we can calculate a handful of terms, say 3, then just ignore the rest and have a very nice estimate. Unfortunately, calculating subsequent terms gets exponentially harder, and entirely new mathematical methods often have to be invented to solve the new class of problems that the next order imposes. Further, these new methods have to be implemented in code, and it must be fast, since we don't have years to wait around for a computer to go through calculations. As a note, this concept of writing things in terms of an infinite series of solutions is called \textit{perturbation theory}.

\subsection{Main Goal(s) for the Project}

For the purposes of my project, I will focus on leading order calculations, meaning I calculate only the first term of the aforementioned infinite series, making my life significantly easier, and providing still decent approximations. If I have time, I will consider how to implement some next-to leading order (NLO) components, but I reckon that will be very challenging. Further, there are few if any programs out there that actually do the entire process from hard scattering to detector simulation all in one. For instance, \textsc{MadGraph5} only focuses on the hard scattering; \textsc{Pythia8} focuses on the parton showering (though it can do hard scattering as well); and \textsc{Geant4} focuses on simulating the detector. So, in my program, I will focus on the hard scattering and parton showering processes, most similarly to \textsc{Pythia8}, and leave out detector simulations and reconstruction algorithms.

The goal is to be able to make distributions of some basic kinematic variables such as (transverse) momentum, energy, and so on, and have it be comparable to the output of \textsc{Pythia8} and others given roughly the same set of inputs. These plots will be of the final-state, stable particles after the hadronization has finished that would then fly through the detector. 

Even though this is technically already an expectation for the project, I plan to have a sophisticated ``housekeeping'' structure including logging, data storage, and general information storage so that the user has access to everything in a readable format whenever they want. Further, as I will discuss a bit more in the implementations section, I will also have some sort of communication with some other framework such that I can draw plots to the screen, like Gnuplot. In principle, the plots and various other generated data should be compatible and comparable with the other simulation suites out there, most specifically to \textsc{Pythia8}. One way this can be done is by storing my results in the Les Houches event file (LHEF) format, which is a universal text-based XML-style file format recognized by most simulation suites in use today (see \cite{LHEFFORMAT} for a super brief description, particularly section 4). These file formats allow direct translation from one simulation suite to another in many cases. For instance, I could generate data corresponding to the hard scattering process with \textsc{MadGraph5} in the LHEF format, then pass it directly to \textsc{Pythia8} to do the parton showering simulation. My point is that one of my goals is to have such a file be the output of my program, among various plots and such.



%%% Local Variables:
%%% mode: LaTeX
%%% TeX-master: "../../ProjectProposal"
%%% End:

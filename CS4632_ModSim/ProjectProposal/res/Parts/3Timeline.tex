\section{Timeline}\label{sec:Timeline}

I reckon that the initial phase of literature review and model formulation will take a substantial amount of time considering the relative complexity of the process I intend to model and simulate. As I mentioned above, The relevent documentation is very extensive, and it will take time to extract only the relevant components that are feasible to implement in this project. As a rough estimate, this will take a few weeks to complete.

In the mean time, I plan on implementing all of my so-called ``housekeeping'' functionality, since it is largely independent of how I choose to implement the model. I have already completed a system for logging information to the console and a dedicated log file, whose purpose will be mostly for debugging at first, but once the simulation has been implemented it will serve as an extensive record of the steps the simulation took during its execution. By the time the literature review is done, I will plan on having the data storage and Gnuplot interface completed, having tested it with fake data. This will have to be a bit more fluid though, since it will have to account for the actual output of the simulation and which variables I end up choosing to implement.

Having this implemented early will be a huge help, particularly the debugging/logging, as I move forward into implementing the actual physics simulation, since it will hopefully allow me to largely avoid some nasty bugs by being able to constantly visualize what my program is doing. The actual implementation of the model, once I have it designed, should hopefully be very straightforward. Several mathematical routines that I imagine I will need to use I have already programmed at some point in the past (though in either Python or C), and if not, I have extensive resources regarding how to implement them.


%%% Local Variables:
%%% mode: LaTeX
%%% TeX-master: "../../ProjectProposal"
%%% End:

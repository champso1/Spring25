\section*{Abstract}

Event generators and simulations of proton-proton collisions in high-energy/particle physics are an integral component in the advancement of our knowledge of the fundamentals of the universe. They and the theoretical frameworks they are based on provide confirmation for what is observed in experiement and provide a means to make predictions once the framework and models are validated, providing a direction for future experiments to go in to make discoveries. The monumental discovery of the Higgs boson in 2012 at the Large Hadron Collider (LHC) in Geneva, Switzerland, was directly fueled by results from simulations. I plan to implement a simplified version of such an event generator simulation, focusing only on leading order results, in which I will model two subprocesses: the hard scattering and parton showering processes. Considerations of quantum mechanical phenomena such as quantum chromodynamics (QCD) are required to implement an accurate simulation. I outline the tools I plan to use, including \textsc{LHAPDF}, a tool to extract parton distributions data, and Gnuplot for plotting data, all of which will be done in C++. I describe the basic model structure including the representation of events and particles via their own classes, along with the implementation of the required mathematical subroutines. I then give a brief timeline of when these things are expected to be completed.

%%% Local Variables:
%%% mode: LaTeX
%%% TeX-master: "../../ProjectProposal"
%%% End:
